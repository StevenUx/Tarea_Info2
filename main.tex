\documentclass{article}
\usepackage[utf8]{inputenc}
\usepackage[spanish]{babel}
\usepackage{listings}
\usepackage{graphicx}
\graphicspath{ {images/} }
\usepackage{cite}

\begin{document}

\begin{titlepage}
    \begin{center}
        \vspace*{1cm}
            
        \Huge
        \textbf{Mover un elemento de una posición A a una una posición B}
            
        \vspace{0.5cm}
        \LARGE
        Por
            
        \vspace{1.5cm}
            
        \textbf{Brayan Estiben Gomez Carmona}
            
        \vfill
            
        \vspace{0.8cm}
            
        \Large
        Despartamento de Ingeniería Electrónica y Telecomunicaciones\\
        Universidad de Antioquia\\
        Medellín\\
        Marzo de 2021
            
    \end{center}
\end{titlepage}
\newpage

            
   \Huge
        \textbf{Pasos para realizar la tarea}
 \section{Tome la hoja con una mano y levantela}
\section{Situe la hoja sobre la mesa o superficie a un costado de las tarjetas}
\section{Tome ambas tarjetas con una mano y alineelas ubicando las yemas de los dedos meñique, anular y medio en uno de los costados largos de ambas tarjetas, el pulgar en el otro lado largo y el indice en el extremo corto cercano, este será el extremo corto superior.}
\section{Sin perder la posicion de los dedos previa, apoye el extremo corto inferior de las tarjetas en un area central de la hoja.}
\section{Ubique el dedo medio en la esquina superior de ese mismo lado, despegue los dedos meñique, anular y pulgar de las tarjetas.}
\section{Con el dedo indice aplicará una fuerza suave y constante a las tarjetas para que no resbalen de la hoja.}
\newpage
\section{Ponga los dedos pulgar y anular en los laterales largos de la tarjeta próxima a la palma de su mano que son cercanos a estos dedos. }

\section{Incline un poco ambas tarjetas en direccion de la palma de su mano, detengase y comience a deslizar suavemente la tarjeta en donde tiene sus dedos anular y pulgar también en esta dirección. Deberá ayudarse de los dedos indice y medio para evitar que la tarjeta resbale. Detengase cuando haya formado una piramide cuya base no sea ni muy ancha ni muy estrecha.}
\section{Una vez lograda la posicion anterior, trate de equilibrar ambas tarjetas y cuando lo consiga separe todos sus dedos suavemente sin perturbar el equilibrio de las tarjetas.}





\end{document}
